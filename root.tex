%%%%%%%%%%%%%%%%%%%%%%%%%%%%%%%%%%%%%%%%%%%%%%%%%%%%%%%%%%%%%%%%%%%%%%%%%%%%%%%%
%2345678901234567890123456789012345678901234567890123456789012345678901234567890
%        1         2         3         4         5         6         7         8

\documentclass[letterpaper, 10 pt, conference]{ieeeconf}  % Comment this line out if you need a4paper

%\documentclass[a4paper, 10pt, conference]{ieeeconf}      % Use this line for a4 paper

\IEEEoverridecommandlockouts                              % This command is only needed if 
                                                          % you want to use the \thanks command

\overrideIEEEmargins                                      % Needed to meet printer requirements.

% See the \addtolength command later in the file to balance the column lengths
% on the last page of the document

% The following packages can be found on http:\\www.ctan.org
%\usepackage{graphics} % for pdf, bitmapped graphics files
%\usepackage{epsfig} % for postscript graphics files
%\usepackage{mathptmx} % assumes new font selection scheme installed
%\usepackage{times} % assumes new font selection scheme installed
%\usepackage{amsmath} % assumes amsmath package installed
%\usepackage{amssymb}  % assumes amsmath package installed

\title{\LARGE \bf
Exoskeletons
}


\author{Aydin Tekin$^{1}$% <-this % stops a space
\thanks{*This work was not supported by any organization}% <-this % stops a space
\thanks{$^{1}$Aydin Tekin is student at the Karlsruhe Institute of Technology, 76131 Karlsruhe, Germany
        {\tt\small aydin.tekin@student.kit.edu}}%
}


\begin{document}



\maketitle
\thispagestyle{empty}
\pagestyle{empty}


%%%%%%%%%%%%%%%%%%%%%%%%%%%%%%%%%%%%%%%%%%%%%%%%%%%%%%%%%%%%%%%%%%%%%%%%%%%%%%%%
\begin{abstract}

In the past twenty years the research about exoskeleton made many large steps towards the vision of a durable, natural and user-friendly exoskeleton, assisting humans in many aspects. The idea of increasing human strength or fixing weaknesses of human nature led to great efforts in making these dreams true. In this paper the current "state of art" and their key functionalities of these systems will be presented.

\end{abstract}


%%%%%%%%%%%%%%%%%%%%%%%%%%%%%%%%%%%%%%%%%%%%%%%%%%%%%%%%%%%%%%%%%%%%%%%%%%%%%%%%
\section{INTRODUCTION}

One way how an exoskeletons is defined, can be described with the following sentence:
"A [...] structure, [...], that provides protection or support for an organism."
The use-cases of an exoskeleton can be divided into three main sectors: military, medicine and industry.\newline
The biggest of these three sectors is the military; it could be seen as the first promoter of developing modern exoskeletons because in the early 60's the US Department of Defence initiated the research on an exoskeleton with the vision to increase the strength and durability of american soldiers. General Electric then presented their first protoype "Hardiman", which weighted more than 750kg and could lift loads up to 250 kg. The fact that the engineers just could get one arm working and it was very heavy/inefficient (could only lift a third of its own weight), the department stopped the coorperation and General Electric dropped that project. Years later DARPA (Defense Advanced Research Projects Agency) started cooperations with several american universities to intensify the research on exoskeletons for military purposes leading on prototypes like BLEEX (Berkeley Lower Extremity EXoskeleton). Private companies understood the brigth future of this sector and bought the most licenses of these prototypes and tweakend them in their internal research centers. The HULC (Human Universal Load Carrier), being actively developed and produced by Lockheed Martin, is an excellent example for these kind of commercialised systems.\newline
The medical sector is the second biggest one; the first prothesis was crafted by the egyptians 600 BC representing a wooden toe. After hundreds of years these kind of proteses were continually enhanced but still were mostly static aritificials. After the past two world wars the demand of protheses for limbs increased dramatically, so researchers started to work on dynamic systems to simulate the human limbs. Also the demography of industrial countries changed, affected by good medical care and lower mortality rate leading to a growing percentage of elderly and disabled people. This forced the medicine to create systems to support and ensure a certain quality of life  for these people.\newline
Modern exoskeletons are designed to be user-friendly, intuitive and safe. The acceptance of the system from the eye of the user should be increased with an high amount of comfortability, agility and usability. 


\section{STATE OF THE ART}

\subsection{Lower body exoskeletons}

First, confirm that you have the correct template for your paper size. This template has been tailored for output on the US-letter paper size. 
It may be used for A4 paper size if the paper size setting is suitably modified.

\subsection{Upper body exoskeletons}

The template is used to format your paper and style the text. All margins, column widths, line spaces, and text fonts are prescribed; please do not alter them. You may note peculiarities. For example, the head margin in this template measures proportionately more than is customary. This measurement and others are deliberate, using specifications that anticipate your paper as one part of the entire proceedings, and not as an independent document. Please do not revise any of the current designations

\subsection{Full body exoskeletons}
 
Test in this test test test test test test test 
test test test test test test test test

\section{DISSCUSION}

TEST

\section{CONCLUSION}

TEST

\addtolength{\textheight}{-12cm}   % This command serves to balance the column lengths
                                  % on the last page of the document manually. It shortens
                                  % the textheight of the last page by a suitable amount.
                                  % This command does not take effect until the next page
                                  % so it should come on the page before the last. Make
                                  % sure that you do not shorten the textheight too much.

%%%%%%%%%%%%%%%%%%%%%%%%%%%%%%%%%%%%%%%%%%%%%%%%%%%%%%%%%%%%%%%%%%%%%%%%%%%%%%%%



%%%%%%%%%%%%%%%%%%%%%%%%%%%%%%%%%%%%%%%%%%%%%%%%%%%%%%%%%%%%%%%%%%%%%%%%%%%%%%%%



%%%%%%%%%%%%%%%%%%%%%%%%%%%%%%%%%%%%%%%%%%%%%%%%%%%%%%%%%%%%%%%%%%%%%%%%%%%%%%%%


\begin{thebibliography}{99}

\bibitem{c1} G. O. Young, �Synthetic structure of industrial plastics (Book style with paper title and editor),� 	in Plastics, 2nd ed. vol. 3, J. Peters, Ed.  New York: McGraw-Hill, 1964, pp. 15�64.
\bibitem{c2} W.-K. Chen, Linear Networks and Systems (Book style).	Belmont, CA: Wadsworth, 1993, pp. 123�135.
\bibitem{c3} H. Poor, An Introduction to Signal Detection and Estimation.   New York: Springer-Verlag, 1985, ch. 4.
\bibitem{c4} B. Smith, �An approach to graphs of linear forms (Unpublished work style),� unpublished.
\bibitem{c5} E. H. Miller, �A note on reflector arrays (Periodical style�Accepted for publication),� IEEE Trans. Antennas Propagat., to be publised.
\bibitem{c6} J. Wang, �Fundamentals of erbium-doped fiber amplifiers arrays (Periodical style�Submitted for publication),� IEEE J. Quantum Electron., submitted for publication.
\bibitem{c7} C. J. Kaufman, Rocky Mountain Research Lab., Boulder, CO, private communication, May 1995.
\bibitem{c8} Y. Yorozu, M. Hirano, K. Oka, and Y. Tagawa, �Electron spectroscopy studies on magneto-optical media and plastic substrate interfaces(Translation Journals style),� IEEE Transl. J. Magn.Jpn., vol. 2, Aug. 1987, pp. 740�741 [Dig. 9th Annu. Conf. Magnetics Japan, 1982, p. 301].
\bibitem{c9} M. Young, The Techincal Writers Handbook.  Mill Valley, CA: University Science, 1989.
\bibitem{c10} J. U. Duncombe, �Infrared navigation�Part I: An assessment of feasibility (Periodical style),� IEEE Trans. Electron Devices, vol. ED-11, pp. 34�39, Jan. 1959.
\bibitem{c11} S. Chen, B. Mulgrew, and P. M. Grant, �A clustering technique for digital communications channel equalization using radial basis function networks,� IEEE Trans. Neural Networks, vol. 4, pp. 570�578, July 1993.
\bibitem{c12} R. W. Lucky, �Automatic equalization for digital communication,� Bell Syst. Tech. J., vol. 44, no. 4, pp. 547�588, Apr. 1965.
\bibitem{c13} S. P. Bingulac, �On the compatibility of adaptive controllers (Published Conference Proceedings style),� in Proc. 4th Annu. Allerton Conf. Circuits and Systems Theory, New York, 1994, pp. 8�16.
\bibitem{c14} G. R. Faulhaber, �Design of service systems with priority reservation,� in Conf. Rec. 1995 IEEE Int. Conf. Communications, pp. 3�8.
\bibitem{c15} W. D. Doyle, �Magnetization reversal in films with biaxial anisotropy,� in 1987 Proc. INTERMAG Conf., pp. 2.2-1�2.2-6.
\bibitem{c16} G. W. Juette and L. E. Zeffanella, �Radio noise currents n short sections on bundle conductors (Presented Conference Paper style),� presented at the IEEE Summer power Meeting, Dallas, TX, June 22�27, 1990, Paper 90 SM 690-0 PWRS.
\bibitem{c17} J. G. Kreifeldt, �An analysis of surface-detected EMG as an amplitude-modulated noise,� presented at the 1989 Int. Conf. Medicine and Biological Engineering, Chicago, IL.
\bibitem{c18} J. Williams, �Narrow-band analyzer (Thesis or Dissertation style),� Ph.D. dissertation, Dept. Elect. Eng., Harvard Univ., Cambridge, MA, 1993. 
\bibitem{c19} N. Kawasaki, �Parametric study of thermal and chemical nonequilibrium nozzle flow,� M.S. thesis, Dept. Electron. Eng., Osaka Univ., Osaka, Japan, 1993.
\bibitem{c20} J. P. Wilkinson, �Nonlinear resonant circuit devices (Patent style),� U.S. Patent 3 624 12, July 16, 1990. 






\end{thebibliography}




\end{document}
