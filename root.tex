%%%%%%%%%%%%%%%%%%%%%%%%%%%%%%%%%%%%%%%%%%%%%%%%%%%%%%%%%%%%%%%%%%%%%%%%%%%%%%%%
%2345678901234567890123456789012345678901234567890123456789012345678901234567890
%        1         2         3         4         5         6         7         8

\documentclass[letterpaper, 10 pt, conference]{ieeeconf}  % Comment this line out if you need a4paper

%\documentclass[a4paper, 10pt, conference]{ieeeconf}      % Use this line for a4 paper

\IEEEoverridecommandlockouts                              % This command is only needed if 
                                                          % you want to use the \thanks command

\overrideIEEEmargins                                      % Needed to meet printer requirements.

% See the \addtolength command later in the file to balance the column lengths
% on the last page of the document

% The following packages can be found on http:\\www.ctan.org
\usepackage{float}
\usepackage{graphicx} % for pdf, bitmapped graphics files
%\usepackage{epsfig} % for postscript graphics files
%\usepackage{mathptmx} % assumes new font selection scheme installed
%\usepackage{times} % assumes new font selection scheme installed
%\usepackage{amsmath} % assumes amsmath package installed
%\usepackage{amssymb}  % assumes amsmath package installed

\title{\LARGE \bf
Exoskeletons
}


\author{Aydin Tekin$^{1}$% <-this % stops a space
\thanks{*This work was not supported by any organization}% <-this % stops a space
\thanks{$^{1}$Aydin Tekin is student at the Karlsruhe Institute of Technology, 76131 Karlsruhe, Germany
        {\tt\small aydin.tekin@student.kit.edu}}%
}


\begin{document}



\maketitle
\thispagestyle{empty}
\pagestyle{empty}


%%%%%%%%%%%%%%%%%%%%%%%%%%%%%%%%%%%%%%%%%%%%%%%%%%%%%%%%%%%%%%%%%%%%%%%%%%%%%%%%
\begin{abstract}

In the past twenty years the research about exoskeleton made many large steps towards the vision of a durable, natural and user-friendly exoskeleton, assisting humans in many aspects. The idea of increasing human strength or fixing weaknesses of human nature led to great efforts in making these dreams true. In this paper the current "state of art" and their key functionalities of these systems will be presented.

\end{abstract}


%%%%%%%%%%%%%%%%%%%%%%%%%%%%%%%%%%%%%%%%%%%%%%%%%%%%%%%%%%%%%%%%%%%%%%%%%%%%%%%%
\section{INTRODUCTION}

One way how an exoskeletons is defined, can be described with the following sentence:
"A [...] structure, [...], that provides protection or support for an organism."
The use-cases of an exoskeleton can be divided into three main sectors: military, medicine and industry.\newline
The biggest of these three sectors is the military; it could be seen as the first promoter of developing modern exoskeletons because in the early 60's the US Department of Defence initiated the research on an exoskeleton with the vision to increase the strength and durability of american soldiers. General Electric then presented their first protoype "Hardiman", which weighted more than 750kg and could lift loads up to 250 kg. The fact that the engineers just could get one arm working and it was very heavy/inefficient (could only lift a third of its own weight), the department stopped the coorperation and General Electric dropped that project. Years later DARPA (Defense Advanced Research Projects Agency) started cooperations with several american universities to intensify the research on exoskeletons for military purposes leading on prototypes like BLEEX (Berkeley Lower Extremity EXoskeleton). Private companies understood the brigth future of this sector and bought the most licenses of these prototypes and tweakend them in their internal research centers. The HULC (Human Universal Load Carrier), being actively developed and produced by Lockheed Martin, is an excellent example for these kind of commercialised systems.\newline
The medical sector is the second biggest one; the first prothesis was crafted by the egyptians 600 BC representing a wooden toe. After hundreds of years these kind of proteses were continually enhanced but still were mostly static aritificials. After the past two world wars the demand of protheses for limbs increased dramatically, so researchers started to work on dynamic systems to simulate the human limbs. Also the demography of industrial countries changed, affected by good medical care and lower mortality rate leading to a growing percentage of elderly and disabled people. This forced the medicine to create systems to support and ensure a certain quality of life  for these people.\newline
Modern exoskeletons are designed to be user-friendly, intuitive and safe. The acceptance of the system from the eye of the user should be increased with an high amount of comfortability, agility and usability. 



\section{STATE OF THE ART}

The different systems presented in this paper can be divided into three general types; lower body-, upper body- and full body exoskeletons.

%%%%%%%%%%%%%%%%%%%%%%%%%%%%%%%%%%%%%%%%%%%%%%%%%%%%%%%%%%%%%%%%%%%%%%%%%%%%%%%%

Lower body exoskeletons are mostly designed for the medical sector. Mostly its used for rehabilitation of people who have difficulties with walking (e.g. caused by long-term *Bett gefesselt sein* or by vestibular disorder). Some of these systems even go a step further and try to give disabled people some amount of their mobility back.

%%%%%%%%%%%%%%%%%%%%%%%%%%%%%%%%%%%%%%%%%%%%%%%%%%%%%%%%%%%%%%%%%%%%%%%%%%%%%%%%

Upper body exoskeletons have different application areas; the XOS 2 for example is planned to be used on military operations, whereas other systems aim to be used in the industry. They could be used in the entertainment industry as an addition to a VR-headset to create a even more authentic virtual reality or as an remote hand in the chemical industry to work with unsafe materials.

%%%%%%%%%%%%%%%%%%%%%%%%%%%%%%%%%%%%%%%%%%%%%%%%%%%%%%%%%%%%%%%%%%%%%%%%%%%%%%%%

The last of the three types are the full body exoskeletons, assisting the whole body. They are used mainly to lift up and transport heavy objects; for example the HULC (Human Universal Load Carrier) helps soldiers to carry more supply on the battlefield. Other systems assist people in their daily life or work.

%%%%%%%%%%%%%%%%%%%%%%%%%%%%%%%%%%%%%%%%%%%%%%%%%%%%%%%%%%%%%%%%%%%%%%%%%%%%%%%%

Altough the following systems were initially designed for a special purpose, due to continous developement, some cant be fit into just one of the mentioned types. For example the HAL (Hybring Assistive Limb) was firstly developed as lower body exoskeleton but after the second generation it was enhanced with an upper body component resulting in a full body solution. So the divison of the exoskeletons are not strict and the affinity of them might be overlap with other types.

\subsection{Lower body exoskeletons}

E-Legs was originally created by Berkeley Bionics in the USA which was later renamed to Ekso Bionics. Also the exoskeleton was relabeled to Ekso. Powered with a battery, attached at the back of the exoskeleton, it can run up to 6 hours. It is used for rehabilitation purposes, mostly for people with walking disabilities. With a self-learing algorithm adapted system control, user have to train several weeks daily, until they can independantly stand up from their wheel-chair and walk freely. Even if the system is still slow and the walking processes seems little stockingly, it gives wheel-chair confined people a great amount of mobility and freedom. This system was commercialed several years ago and is currently used/tested in rehabilitation centers but also can be bought for circa 100.000 US-Dollar. \newline

%%%%%%%%%%%%%%%%%%%%%%%%%%%%%%%%%%%%%%%%%%%%%%%%%%%%%%%%%%%%%%%%%%%%%%%%%%%%%%%%

\begin{figure}[H]
  \centering
    \includegraphics[width=0.5\textwidth]{img/elegs}
  \caption{E-Legs/Ekso by Ekso Bionics, USA}
\end{figure}

%%%%%%%%%%%%%%%%%%%%%%%%%%%%%%%%%%%%%%%%%%%%%%%%%%%%%%%%%%%%%%%%%%%%%%%%%%%%%%%%

LOPES (LOwer Powered ExoSkeleton) is being developed by the University of Twente in the Netherlands and is a stationary exoskeleton. Used also for rehabilitation like Ekso, the main difference is that this system is tethered and mostly controlled by a second operator sitting behind the computer attached to the exoskeleton. The target group are not disabled people but patients who were lost muscle mass after a long stay in hospital or people suffering vestibular disorder. The patients can be personally trained with the system which collectes data on the fly. Depending on the data, the operator can then tune the system to give the user more freedom or assist him more with the walking processes. Additionally having 8 degrees of freedom on a the legs, LOPES grants full flexibility and very natural walking flow for the user.\newpage

%%%%%%%%%%%%%%%%%%%%%%%%%%%%%%%%%%%%%%%%%%%%%%%%%%%%%%%%%%%%%%%%%%%%%%%%%%%%%%%%

\begin{figure}[H]
  \centering
    \includegraphics[scale=1.1]{img/lopes}
  \caption{LOPES by University of Twente, Netherlands}
\end{figure}

%%%%%%%%%%%%%%%%%%%%%%%%%%%%%%%%%%%%%%%%%%%%%%%%%%%%%%%%%%%%%%%%%%%%%%%%%%%%%%%%

The Honda Walking Assist is a lightweight lower extremity exoskeleton developed and maintained by Honda Motor Co., Ltd. in Japan. With a total weight of 2.6 kg (including battery) it's a very mobile device running up to 4 hours with one charge. The battery is also very small and is attached at the back of the device like in ELegs. Originally it was designed to be used by workers in Honda production centers to assist them in their work, it now has even more application areas. It can be used in daily life but also for rehabilitation or for elderly people having difficulties in for example walking upstairs. After 16 years of developement and testing it is used in about 50 hospitals all around japan and can be leased by companies.

%%%%%%%%%%%%%%%%%%%%%%%%%%%%%%%%%%%%%%%%%%%%%%%%%%%%%%%%%%%%%%%%%%%%%%%%%%%%%%%%

\begin{figure}[H]
  \centering
    \includegraphics[scale=0.5]{img/honda}
  \caption{Honda Walking Assists by Honda Motor, Japan}
\end{figure}
\newpage

%%%%%%%%%%%%%%%%%%%%%%%%%%%%%%%%%%%%%%%%%%%%%%%%%%%%%%%%%%%%%%%%%%%%%%%%%%%%%%%%

\subsection{Upper body exoskeletons}

ExoHand is a commercial product distributed by Festo Corporate in Germany. Like LOPES its a tethered system and uses only a component for the hand. One of its key-features is the force-feedback system, which enables the system new use-cases. Initially it was designed for industrial use; it would be used as a remote interface to work with risky materials offering the operators maximum freedom grades by controling a five-fingered remote hand. Later on, the developers saw potential in the force-feedback system and started to create virtual environments in which you could operate with the hand. In addition to a virtual reality headset you could create realistic games where the user doesnt just see a virtual world but also can feel and operate in it. In its very extendible and modular nature, researchers created a brain computer inface which paved the way to medical use-cases. Patients who had an apoplexy can be trained with it, in this way the hand to brain communication, which might been severly violated, can be partly recovered.

%%%%%%%%%%%%%%%%%%%%%%%%%%%%%%%%%%%%%%%%%%%%%%%%%%%%%%%%%%%%%%%%%%%%%%%%%%%%%%%%

\begin{figure}[H]
  \centering
    \includegraphics[width=0.5\textwidth]{img/exohand}
  \caption{ExoHand by Festo, Germany}
\end{figure}

%%%%%%%%%%%%%%%%%%%%%%%%%%%%%%%%%%%%%%%%%%%%%%%%%%%%%%%%%%%%%%%%%%%%%%%%%%%%%%%%

XOS 2 initially developed by Sarcos was later on bought up by Raytheon, USA. In 2000 DARPA requested for military exoskeleton and XOS could enfore against 14 other exemplars. It is fully designed for military use and contains two components; the main upper extremity component and the lower extremity component. The exoskeleton gives the user enough assistance to lift up to 90 kg and easily break through wooden boards. It's said that one soldier wearing the XOS can replace 3 other soldiers in the battle field. But the XOS has beneath its highly agile functionality a big disadvantage; its still tethered to a station because Raytheon still couldnt find a mobile power source which can provide enough power for the suit. The US Army is still doing tests and its expected that it enters military service by 2015, altough its not expected to have a working untethered version until 2020.

\newpage

\begin{figure}[H]
  \centering
    \includegraphics[width=0.4\textwidth]{img/xos2}
  \caption{XOS 2 by Sarcos/Raytheon, USA}
\end{figure}

\subsection{Full body exoskeletons}

-BLEEX

%%%%%%%%%%%%%%%%%%%%%%%%%%%%%%%%%%%%%%%%%%%%%%%%%%%%%%%%%%%%%%%%%%%%%%%%%%%%%%%%

\begin{figure}[H]
  \centering
    \includegraphics[width=0.4\textwidth]{img/bleex}
  \caption{BLEEX by University of California, USA}
\end{figure}

%%%%%%%%%%%%%%%%%%%%%%%%%%%%%%%%%%%%%%%%%%%%%%%%%%%%%%%%%%%%%%%%%%%%%%%%%%%%%%%%

-HULC

%%%%%%%%%%%%%%%%%%%%%%%%%%%%%%%%%%%%%%%%%%%%%%%%%%%%%%%%%%%%%%%%%%%%%%%%%%%%%%%%

\begin{figure}[H]
  \centering
    \includegraphics[width=0.4\textwidth]{img/hulc}
  \caption{HULC by Lockheed Martin, USA}
\end{figure}

%%%%%%%%%%%%%%%%%%%%%%%%%%%%%%%%%%%%%%%%%%%%%%%%%%%%%%%%%%%%%%%%%%%%%%%%%%%%%%%%

-HAL

%%%%%%%%%%%%%%%%%%%%%%%%%%%%%%%%%%%%%%%%%%%%%%%%%%%%%%%%%%%%%%%%%%%%%%%%%%%%%%%%

\begin{figure}[H]
  \centering
    \includegraphics[width=0.4\textwidth]{img/hal}
  \caption{HAL-5 by Cyberdyne, Japan}
\end{figure}

%%%%%%%%%%%%%%%%%%%%%%%%%%%%%%%%%%%%%%%%%%%%%%%%%%%%%%%%%%%%%%%%%%%%%%%%%%%%%%%%

\section{DISSCUSION}

Altough the presented systems are very advanced, there are still difficulties the researchers have to face developing efficient systems. 

%%%%%%%%%%%%%%%%%%%%%%%%%%%%%%%%%%%%%%%%%%%%%%%%%%%%%%%%%%%%%%%%%%%%%%%%%%%%%%%%

The sensor technology is still not as advanced as we think. Many systems still dont use any active sensors to to obtain human body signals, instead passive sensors measure the movement initiated by the human then try to estimate the future movement and then react to this. This has the advantage that you can easily put on the exoskeleton without having to place sensors accuratetly on defined spots on your body. But the kind of sensor system is very inaccurate and reacts pretty slowly to human interaction, which may restrict the agility of the user. Some systems use Electromyography, detecting eletrical potential generated by muscels, using EMG-Sensors on the skin of the user to analyse the movement of him. Even if we get much better results, they are still very inaccurate because we cant "dock" directly on the muscle cells. Additionally it is very difficult for a normal user to put on these kind of exoskeletons because you need an experienced assistant to put the sensors to the right spot on the body. Any deviantions could have an immense influence of the algorithm used for motion detection. Currently researchers are trying to optimize the existing sensors but also are working on other ways to obtain human signals like using EMG-needles.

%%%%%%%%%%%%%%%%%%%%%%%%%%%%%%%%%%%%%%%%%%%%%%%%%%%%%%%%%%%%%%%%%%%%%%%%%%%%%%%%

Security is also seen as an important feature in developing such systems. People who use this system must be able to trust them and the system should be "secure" otherwise the user could get injured in many ways. In the worst case the exoskeleton may wrench extremities of humans or could suddenly shutdown leaving the user back with a heavy load. That's why the researcher try to define guidelines for secure systems; one of them is ISO 13485. It covers different aspects of secure medical systems like corrective and preventive actions or risk management. HAL-5 is the first exoskeleton who has this certification and other systems are expected to follow.

%%%%%%%%%%%%%%%%%%%%%%%%%%%%%%%%%%%%%%%%%%%%%%%%%%%%%%%%%%%%%%%%%%%%%%%%%%%%%%%%

Most focus on the recent work is the question which power supply should be used and how it can be optimized. Some systems like XOS dont even use any autonomous power supplies, they are attached with wires to a station. This gives a full electrical durability but they can only be used stationary. For independant exoskeletons there are currently two different types of power supplies; the first one is using a battery which is attached to the exoskeleton. This solution has a very low capacity, so the systems can only work for a few hours which might be a very short length of time, depending on the use-case. Also batteries are heavy, recharging takes too much time and the durability of a battery decreases over time. Fuels are an alternative which can trump with a high and long-lasting energy output resulting in operating time of several days. Refilling can be done very fast and you dont have to find fixed stations to refill the energy source. But it also has its downsides, it's very noisy to use and very insecure: the content is highly flammable. The HAL-5 uses a highly optimized battery and can work up to 5 hours with one charge, using it in an medical environment there is no way to use noisy and smoking energy sources. HULC is planned to use a fuel based power supply, its important for these kind of military-used exoskeletons to have a large operation time. Until there is no other good alternative, one of the mentioned types have to be used and the disadvantes have to be accepted.
 

\section{CONCLUSION}

The human body has flaws and limitations which will and have to be fixed with the increasing technical possibilites we have. With exoskeletons, it seems like a solution for this problem is found, so the work on improving must be intensified.

%%%%%%%%%%%%%%%%%%%%%%%%%%%%%%%%%%%%%%%%%%%%%%%%%%%%%%%%%%%%%%%%%%%%%%%%%%%%%%%%

Companies understood the importance of this sector and started commercializing products, which to then just were seen as objects in research. Evermore exoskeletons are made accessible to the general public this way, even if the prices are still too high for the most people to buy them.

%%%%%%%%%%%%%%%%%%%%%%%%%%%%%%%%%%%%%%%%%%%%%%%%%%%%%%%%%%%%%%%%%%%%%%%%%%%%%%%%


\begin{thebibliography}{99}

\bibitem{c1} Brown, P., Jones, D., Singh, S. K., and Rosen, J. M., The exoskeleton glove for control of paralyzed hands.In Proceedings of the IEEE 
International Conference on Robotics and Automation, Atlanta, GA, USA, 1993, vol. 1,pp. 642-647.
\bibitem{c2} Robert Bogue, Exoskeletons and robotic prosthetics: a review of recent developments, 2009
\bibitem{c3} Dollar A.M., Herr H., Lower extremity exoskeletons and active orthoses: challenges and state-of-the-art, 2008
\bibitem{c4} Guizzo E., Goldstein H., The rise of the body bots [robotic exoskeletons], 2005
\bibitem{c5} Shigeo Tanabe, Satoshi Hirano, Eiichi Saitoh, Wearable Power-Assist Locomotor (WPAL) for supporting upright walking in persons with paraplegia, 2013
\bibitem{c6} Zoss A., Kazerooni H., Chu A., On the mechanical design of the Berkeley Lower Extremity Exoskeleton (BLEEX), 2005 
\bibitem{c7} Kazerooni H., Exoskeletons for human power augmentation, 2005
\bibitem{c8} Naruse, K., Kawai, S., Yokoi, H., and Kakazu, Y. Devel-opment of wearable exoskeleton power assist sys-tem for lower back support. In Proceedings of theIEEE/RSJ International Conference on Intelligent Robotsand Systems, Las Vegas, Nevada, USA, 2003, vol. 3,pp. 3630-3635.
\bibitem{c9} Pratt J.E., Krupp B.T., Morse C.J., Collins S.H., The RoboKnee: an exoskeleton for enhancing strength and endurance during walking, 2004
\bibitem{c10} Veneman J.F., Kruidhof R., Hekman E.E.G., Ekkelenkamp R., Design and Evaluation of the LOPES Exoskeleton Robot for Interactive Gait Rehabilitation, 2007
\bibitem{c11} Tsagarakis N. G., Caldwell Darwin G., Development and Control of a ?Soft-Actuated? Exoskeleton for Use in Physiotherapy and Training, 2003
\bibitem{c12} Banala S.K., Seok Hun Kim, Agrawal S.K., Scholz J.P., Robot Assisted Gait Training With Active Leg Exoskeleton (ALEX), 2008
\bibitem{c13} Gupta A., O'Malley M.K., Design of a haptic arm exoskeleton for training and rehabilitation, 2006 
\bibitem{c14} Walsh C.J., Paluska D., Pasch K., Grand W., Development of a lightweight, underactuated exoskeleton for load-carrying augmentation, 2006
\bibitem{c15} Koyama T., Yamano I., Takemura K., Maeno T., Multi-fingered exoskeleton haptic device using passive force feedback for dexterous teleoperation, 2002

\end{thebibliography}




\end{document}
